\documentclass[twocolumn]{article}

\usepackage{color}
\usepackage[margin=1in,bottom=1.5in]{geometry}

% Comment out second line to disable.
% Thanks: https://gist.github.com/orbekk/1298622
\newcommand{\todo}[1]{}
\renewcommand{\todo}[1]{{\color{red} TODO: {#1}}}

\newcommand{\secref}[1]{Section~\ref{sec:#1}}
\newcommand{\seclab}[1]{\label{sec:#1}}
\newcommand{\figref}[1]{Figure~\ref{fig:#1}}
\newcommand{\figlab}[1]{\label{fig:#1}}
\newcommand{\tblref}[1]{Table~\ref{tbl:#1}}
\newcommand{\tbllab}[1]{\label{tbl:#1}}

\title{Modeling Image Segmentation as Epidemic Spread in an Interdependent Network}
\author{
  Connor Greenwell, Emory Hufbauer\\
  Computer Science Dept. \\
  University of Kentucky
}
\date{}

\begin{document}

\maketitle

\begin{abstract}
In this document we propose formulating the problem of image segmentation as simulating the propagation of
information/ownership through an interdependent network where the first layer is a lattice representation of an input image
and the second layer is a (planar) overlay graph where each node has weighted edges to pixels in the lattice. We will design
and evaluate a number of propagation models, primarily based on pixel/region similarity metrics. Finally we will compare our
method against classical and current state-of-the-art methods for image segmentation on a variety of segmentation benchmark
datasets. 
\end{abstract}

\section{Introduction}

\todo{
Engaging opening paragraph. 

``Instance segmentation is an important and popular topic in computer vision \cite{newell2017associative, li2017fully,
ren2017end}. Current state of the art uses recurrent neural networks to segment an image into its consituent object parts.
However occlusion is not handled well (is this actually true?). We propose a method based on the endemic spread of information
through an interdependent network.'' 
}

\todo{some light background on image segmentation.}

\todo{some light background on interdependent networks. Why is this problem formulation better?}

\todo{
Summarize the paper into a medium length paragraph (currently just copy/pasted from the abstract.): In this document we propose formulating the problem of image segmentation as simulating the propagation of
information/ownership through an interdependent network where the first layer is a lattice representation of an input image
and the second layer is a (planar) overlay graph where each node has weighted edges to pixels in the lattice. We will design
and evaluate a number of propagation models, primarily based on pixel/region similarity metrics. Finally we will compare our
method against classical and current state-of-the-art methods for image segmentation on a variety of segmentation benchmark
datasets. 
}

The following paper is layed out as follows: In \secref{related} we survey the current state of the art in image segmentation,
as well as methods related to ours. In \secref{approach} we describe our method in detail. In \secref{eval} we describe the
metrics by which we judge our methods and those we compare against. In \secref{data} we describe the benchmark datasets
against which we compare. Finally, in \secref{results} we describe our resulting performance and evaluation.

\section{Related Work}\seclab{related}

\todo{endemic spread in interdpenedent networks}

\todo{image segmentation: classic, modern, and weird methods}
\cite{pei2014saliency} use a Markov-Random-Field on precomputed super pixels to perform image segmentation.
\cite{newell2017associative,li2017fully,ren2017end} present a variety of neural network based approached to instance
segmentation on natural images.

\section{Approach}\seclab{approach}

First we encode an input image as a lattice graph of pixels. A random, planar overlay graph is created and edges linked
between each node and every pixel in the image. The overlay graph has a low number of nodes, representing a prior on the
maximum number of objects that can possibly exist in a photograph. 

We will first create an interdependent network model of an image. The weights of edges between adjacent pixels in the lattice
will represent their degree of similarity to each other. The weights of edges between nodes in the overlay graph will
represent their overlap and estimated potential to belong to the same object. The weights of edges between the overlay and
lattice will represent ownership of pixels by superpixels.

We will then perform a simultaneous, competitive propagation of multiple phenomena through this network using a variety of
models, with the ultimate goal of developing a propagation model with property that, after propagation has completed, the
regions of the lattice affected by each phenomena correspond well to the segments of the image.

\todo{more complete description of planned experiments}

\todo{
A naive baseline method has been developed for us to compare our actual method against. It is based on using DBSCAN
\cite{ester1996density} to cluster and merge superpixels produced by SLIC \cite{achanta2010slic}.
}

\section{Evaluation}\seclab{eval}

The goal of this project is mostly to explore the space and point to future research possibilities. Although the results will
be compared with those of existing algorithms, they are not expected to be competitive with the state of the art. To that end
we will evaluate our performance on a variety of classic image segmentation benchmark datasets, including PASCAL VOC and
MS-COCO, as well as compare our results against existing state-of-the-art segmentation methods. 

\todo{describe metrics here, probably need to restate the formula from Hebert's paper}

\todo{
A number of evaluation metrics have been explored for performance on the task of image segmentation. We have chosen the
Adjusted Rand Score from \cite{unnikrishnan2005measure}  Some other metrics to consider are Mutual Information, FMI, and
Homogeneity scores. These may be included in the final paper if they tell an interesting story but our primary focus will be
on evluating against Hebert's ARS score.
}

\subsection{Datasets}\seclab{data}

\todo{we dont need all 4, trim down a bit probably}

\paragraph{PASCAL VOC} \todo{describe} \cite{Everingham10}.

\paragraph{Cityscapes Dataset} \todo{describe} \cite{cordts2016cityscapes}.

\paragraph{KITTI} \todo{describe} \cite{geiger2012we}

\paragraph{MS COCO} \todo{describe} \cite{lin2014microsoft}.

\section{Results}\seclab{results}

\todo{This section should be a boat load of tables and plots. Hopefully we look good in some of them}

\todo{
Mechanisms for performing hyperparameter optimization have been developed and tested on the naive baseline. This will be
necessary because our final method will likely have a number of tunable hyperparameters and it will be useful to automatically
find the optimal settings for our task.
}

\section{Conclusion}

\todo{restate what we did in more detail than we do in the intro}

\todo{future work}

\bibliographystyle{plain}
\bibliography{refs} 

\end{document}
