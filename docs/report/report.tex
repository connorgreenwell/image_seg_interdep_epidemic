\documentclass[twocolumn]{article}

\title{Modeling Image Segmentation as Epidemic Spread in an Interdependent Network}
\author{Emory Hufbauer, Connor Greenwell}
\date{}

\begin{document}

\maketitle

\begin{abstract}
In this document we propose formulating the problem of image segmentation as simulating the propagation of
information/ownership through an interdependent network where the first layer is a lattice representation of an input image
and the second layer is a (planar) overlay graph where each node has weighted edges to pixels in the lattice. We will design
and evaluate a number of propagation models, primarily based on pixel/region similarity metrics. Finally we will compare our
method against classical and current state-of-the-art methods for image segmentation on a variety of segmentation benchmark
datasets. 
\end{abstract}

\section{Related Work}

\cite{pei2014saliency} use a Markov-Random-Field on precomputed super pixels to perform image segmentation.

\section{Objective}

We will first create an interdependent network model of an image. The weights of edges between adjacent pixels in the lattice
will represent their degree of similarity to each other. The weights of edges between nodes in the overlay graph will
represent their overlap and estimated potential to belong to the same object. The weights of edges between the overlay and
lattice will represent ownership of pixels by superpixels.

We will then perform a simultaneous, competitive propagation of multiple phenomena through this network using a variety of
models, with the ultimate goal of developing a propagation model with property that, after propagation has completed, the
regions of the lattice affected by each phenomena correspond well to the segments of the image.

\section{Inputs}

An image and, potentially, a super-pixelization of that image created using an existing algorithm to be used as an overlay
graph.

\section{Output}

A reasonable segmentation of the image based on a union of superpixels in the overlay graph.

\section{Evaluation}

The goal of this project is mostly to explore the space and point to future research possibilities. Although the results will
be compared with those of existing algorithms, they are not expected to be competitive with the state of the art. To that end
we will evaluate our performance on a variety of classic image segmentation benchmark datasets, including PASCAL VOC and
MS-COCO, as well as compare our results against existing state-of-the-art segmentation methods. 

\bibliographystyle{plain}
\bibliography{refs} 

\end{document}
